\twocolumn[
\maketitle
\begin{@twocolumnfalse}
\begin{abstract}

En esta practica de laboratorio, se midieron curvas de volaje con respecto a la corriente $V(I)$ para medir la resistencia del tungsteno. Se utilizaron dos métodos, uno manual y otro automatizado. El el método manual se encontró que solo se cumplia para valores pequeños de $I$, por lo que se tomó como si cumpliera la ley de Ohmn. Para el método automatizado, se encontró el factor $L/A$ que relaciona la resistividad ($\rho$) con la resistencia ($R$) para así encontrar la temperatura con respecto a la resistencia, obteniendo una relación lineal. Comparando con la literatura, se obtuvó una buena aproximación para el método manual y automático.

\end{abstract}
\end{@twocolumnfalse}
]