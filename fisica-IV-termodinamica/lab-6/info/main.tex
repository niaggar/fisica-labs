\documentclass[10pt]{article}
\usepackage[utf8]{inputenc}
\usepackage{geometry}
\usepackage{graphicx,graphics}
\usepackage{hyperref}
\usepackage{graphicx}
\usepackage{url}
\usepackage{float}
\usepackage{amsmath}
\usepackage{multicol}
\usepackage{amssymb}
\usepackage{multirow}
\usepackage{textcomp}
\usepackage{gensymb}
\usepackage{hhline}
\usepackage{array}
\usepackage{caption}
\usepackage[spanish, es-tabla]{babel}
\usepackage[center]{titlesec}
\usepackage[table,xcdraw]{xcolor}

\geometry{top=2cm, bottom=2cm, left=1.5cm, right=1.5cm}
\spanishdecimal{,}
\setlength{\parskip}{0mm}

\renewcommand{\arraystretch}{1.3}
\renewcommand{\thesection}{\large\Roman{section}}
\renewcommand{\thesubsection}{\large\Alph{subsection}}

\providecommand{\abs}[1]{\lvert#1\rvert}
\title{
    \textbf Determinación experimental del calor específico de metales y el calor latente de fusión del hielo
}
\author{
    \normalsize{
        \emph{Andrés F. Valencia Fonseca}$^{1}$,
        \emph{Nicolás Aguilera García}$^{2}$
    } \\
    \normalsize{
        \emph{Departamento de Física, Universidad del Valle, Cali, Colombia}
    } \\
    \small{$^{1}$2125166, $^{2}$21273030}
}
\date{(\small 13 de mayo de 2023)}

\begin{document}
\maketitle
\begin{abstract}
    El objetivo de este laboratorio es determinar el calor
    específico de varios metales, aluminio, cobre y hierro, y el calor latente de fusión del hielo. Se utilizaron
    métodos de calentamiento y enfriamiento para medir el calor específico de los
    metales, mientras que el método de fusión se empleó para determinar el calor latente del hielo.
    Los resultados experimentales obtenidos para el calor específico de los metales
    fueron consistentes con los valores teóricos de referencia, obteniendo que para el aluminio el calor especifico es
    de  $0.207 , cal/\,g\,^oC$, para el cobre de $0.099 , cal/\,g\,^oC$, y para el
    hierro de $0.117 , cal/\,g\,^oC$, reportando un error porcentual
    en general menor a $7\%$ lo que indica la precisión del método utilizado y la validación
    del marco teorico. Además, se observó que cada metal tenía un
    calor específico único, lo que sugiere diferencias en su capacidad para
    almacenar calor.
    En cuanto al calor latente del hielo, el valor promedio del mismo se calcula
    de $L_f = 75.26 , cal / g$ con una desviación estándar de $6.35$ y un
    porcentaje de error del $5.92 \%$ atribuidos a
    errores experimentales. Los obtenidos experimentales se
    aproximaron al valor aceptado de referencia y destacan su importancia para
    caracterizar la transición de fase del hielo. Estos resultados resaltan la relevancia
    de comprender las propiedades térmicas de los materiales y su aplicación en campos
    como la ingeniería y la ciencia de los materiales.
\end{abstract}

\begin{multicols*}{2}
    \section{\small INTRODUCCIÓN}
    \subsection*{\small Calor especifico de metales}
    El calor específico es una propiedad fundamental de los materiales que
    determina su capacidad para almacenar calor. El
    calor específico varía de un material a otro y se define como la cantidad de
    calor necesaria para elevar la temperatura de una unidad de masa de dicho
    material en una unidad de temperatura, este se relaciona matematicamente como sigue:

    \begin{equation}
        \begin{split}
            Q &= m \, c \, \Delta T \\
            &= m \, c \, (T_2 - T_1)
        \end{split}
        \label{eq:flujo_calor}
    \end{equation}

    Donde $Q$ es el flujo de energia termica (Calor), $m$ es la masa de la
    sustancia, $c$ es la capacidad calorifica de la sustancia y $\Delta T$ es la
    diferencia de temperatura entre el estado inicial y el estado final de la
    sustancia.\\
    El objetivo principal es caracterizar la capacidad de estos
    metales para absorber y liberar calor en función de su masa y temperatura.
    Para ello, se estudiaran 3 materiales en principio desconocnidos, los cuales se identificaran, 
    conociendo su calor especifico, y comparando con los valores registrados de referencia.

    \subsection*{\small Calor latente de fusión del hielo}
    La materia se presenta en diferentes fases, las cuales pueden variar en función
    de las condiciones de presión y temperatura a las que se someta. A este proceso
    se le denomina cambio de fase o estado (siempre y cuando no ocurra un cambio en
    la composición) \cite{calorLatente}. Estos cambios de estado no ocurren de
    forma espontánea y requieren un aporte o liberación de energía, conocida como
    calor latente, que representa la energía calórica necesaria para que una
    sustancia cambie de estado sin que su temperatura varíe durante el proceso.

    La ecuación que modela este proceso en una sustancia es la siguiente
    \cite{calorLatente}:

    \begin{equation}
        Q_f = m \, L_f
        \label{eq:calor_latente}
    \end{equation}

    Donde $Q$ es la cantidad de calor requerida para fundir la sustancia, $m$ es la
    masa de la sustancia y $L_f$ es el calor latente de fusión de la sustancia.

    En este laboratorio se busca determinar el calor latente de fusión del hielo,
    el cual es la energía calórica requerida para que el agua ($H_2O$) cambie de
    fase sólida (hielo) a fase líquida.

    \section{\small METODOLOGÍA}
    Los procesos que se llevaron a cabo en este laboratorio fueron los siguientes:

    \begin{itemize}
        \item \textbf{Calor especifico de metales:}

              El proceso para determinar el calor especifio de los tres diferentes metales
              utilizados como muestras consiste en primeramente determinar la masa de cada
              uno de los metales a utilizar, la masa del calorimetro y la masa del agua que
              estara dentro del calorimetro. Luego de esto, se procede a calentar los metales
              en un baño de agua hirviendo (llegando a temperaturas de $97 \, ^oC$), para
              posteriormente sumergirlos en el calorímetro con agua a temperatura ambiente
              $T_0$.

              Tras esto, se espera a que el sistema se encuentre en equilibrio térmico, y se
              toma la temperatura final del sistema $T_e$. Con estos datos, se procede a
              calcular el calor específico de cada metal. Dado que el calorímetro se
              encuentra aislado térmicamente, se puede afirmar que en el sistema:

              \begin{equation}
                  Q_{ganado} = Q_{cedido}
                  \label{eq:relacion_calor}
              \end{equation}

              Donde $Q_{ganado}$ es el calor ganado por el agua y el calorímetro, y
              $Q_{cedido}$ es el calor cedido por el metal. El calor ganado por el agua, el
              calorímetro y el metal se calcula mediante la ecuación \ref{eq:flujo_calor}.
              Reemplazando y despejado el calor especifico del metal se obtiene la siguiente
              ecuación que permite realizar el calculo mediante los datos obtenidos
              experimentalmente:

              \begin{equation}
                  c_{m} = \frac{M_{agua} \, c_{agua} + M_{cal} \, c_{cal} + M_{agi} \, c_{agi}}{M_{metal} \, (T_{metal} - T_{e})} \cdot (T_{e} - T_{0})
                  \label{eq:calor_especifico}
              \end{equation}

        \item \textbf{Calor latente de fusión del hielo:}

              Para determinar el calor latente de fusión, se utiliza la ecuación
              \ref{eq:calor_latente}, la cual requiere que se determine el calor necesario
              para fundir una cantidad de hielo. Para ello, se utiliza un calorímetro de masa
              $M'$, el cual se encuentra aislado térmicamente y cuya capacidad calórica
              ($c_{cal}$) es conocida. Se agrega una masa de agua $M$ con capacidad
              calorífica $c_{agua}$ y temperatura inicial $T_2$, que se mezcla con una masa
              de hielo $m$ a temperatura inicial $T_0$. Tras la mezcla, se espera un tiempo
              prudencial hasta que la totalidad del hielo se haya fundido completamente, y se
              toma la temperatura $T$ del agua, la cual se encuentra en equilibrio térmico.
              Con estos datos, se procede a calcular el calor latente de fusión del hielo.

              El calculo al igual que en el caso anterior, se realiza mediante la ecuación
              \ref{eq:relacion_calor}, pero teniendo en cuenta que el calor ganado y cedido
              son diferentes.

              Para este caso, el $Q_{ganado}$ es el flujo de calor dado por el hielo y el
              agua resultante del hielo fundido, mientras que $Q_{cedido}$ es el flujo de
              calor dado por el calorímetro y el agua inicialmente contenida en él. Si se
              toma en cuenta la ecuación \ref{eq:flujo_calor}, el calor ganado y cedido
              pueden expresarse como:

              \begin{equation}
                  \begin{split}
                      Q_{ganado} &= m \, c_{agua} \, (T - 0) + m \, L_f + m \, c_{hielo} \, (0 - T_0)\\
                      Q_{cedido} &= (M \, c_{agua} + M' \, c_{cal}) \, (T_2 - T)
                  \end{split}
                  \label{eq:calor_latente_2}
              \end{equation}

              Al reemplazar estos valores en la ecuación \ref{eq:calor_latente_1} y despejar
              el valor buscado ($L_f$), se obtiene:

              \begin{equation}
                  \begin{split}
                      L_f &= A - B \\
                      A &= \frac{(M \, c_{agua} + M' \, c_{cal}) \, (T_2 - T)}{m} \\
                      B &= c_{agua} \, (T - 0) + c_{hielo} \, (0 - T_0)
                  \end{split}
                  \label{eq:calor_latente_3}
              \end{equation}

              De esta forma, con los datos obtenidos en el laboratorio, se determina entonces
              el calor latente de fusión del hielo.
    \end{itemize}

    \section{\small ANÁLISIS Y RESULTADOS}
    \subsection*{\small Calor latente de fusión del hielo}
    Con los resultados obtenidos en el laboratorio, se procede a calcular el calor
    latente de fusión del hielo. Para esto, se tienen además los siguientes datos:

    \begin{itemize}
        \item $M' = 92.8 , g$
        \item $c_{agua} = 1.00 , cal/g \cdot C^{-1}$
        \item $c_{cal} = 0.22 , cal/g \cdot C^{-1}$
        \item $c_{hielo} = 0.50 , cal/g \cdot C^{-1}$
    \end{itemize}

    Los resultados se obtuvieron al realizar el proceso descrito en la metodología
    para tres masas de hielo diferentes, donde para cada masa se tomó una medida a
    alta temperatura (en un rango entre $37 , ^oC$ y $39 , ^oC$) y una a
    temperatura ambiente ($34 , ^oC$), obteniendo un total de 6 medidas. Los
    resultados obtenidos se muestran en la Tabla \ref{tab:calor_latente}.

    \begin{table}[H]
        \centering
        \caption{Resultados obtenidos para el calor latente de fusión del hielo.}
        \begin{tabular}{ccccccc}
            \hline\hline
            \textbf{$M$} & \textbf{$m$} & \textbf{$T_2$} & \textbf{$T_0$} & \textbf{$T$}   & \textbf{$L_f$}     & \textbf{$\% Error$} \\
            \small $[g]$ & \small $[g]$ & \small $[^oC]$ & \small $[^oC]$ & \small $[^oC]$ & \small $[cal / g]$ &                     \\
            \hline\hline
            $147.75$     & $14.31$      & $37$           & $7$            & $28$           & $81.26$            & $1.58$              \\
            $163.13$     & $14.62$      & $24$           & $7$            & $17$           & $74.38$            & $7.02$              \\
            \hline
            $182.91$     & $6.11$       & $39$           & $7$            & $36$           & $67.33$            & $15.83$             \\
            $177.08$     & $6.85$       & $24$           & $7$            & $21$           & $68.99$            & $13.76$             \\
            \hline
            $161.00$     & $5.04$       & $38$           & $7$            & $35$           & $76.49$            & $4.39$              \\
            $224.65$     & $4.80$       & $24$           & $6$            & $22$           & $83.11$            & $3.89$              \\
            \hline
        \end{tabular}
        \caption*{}
        \label{tab:calor_latente}
    \end{table}

    De estos resultados se puede observar que no existe una relación entre la
    certeza del resultado y la temperatura inicial utilizada para el agua. Si se
    revisan los resultados por las masas utilizadas, se obtiene lo siguiente:

    \begin{itemize}
        \item $m \approx 14.6 , g$: Desviación estándar de $4.87$, el valor promedio es de $L_f = 77.82 , cal / g$ y un porcentaje de error del $2.72 \%$.
        \item $m \approx 6.8 , g$: Desviación estándar de $1.17$, el valor promedio es de $L_f = 68.16 , cal / g$ y un porcentaje de error del $14.79 \%$.
        \item $m \approx 5.0 , g$: Desviación estándar de $4.68$, el valor promedio es de $L_f = 79.80 , cal / g$ y un porcentaje de error del $0.25 \%$.
    \end{itemize}

    Se puede observar que los resultados concuerdan con el valor teórico, siendo
    bastante más preciso el obtenido para $m \approx 5.0 , g$. Sin embargo, al
    igual que en el caso de la temperatura, no se observa una relación entre la
    masa y la certeza del resultado.

    En cuanto al valor promedio del calor latente de fusión del hielo, se obtiene
    que es de $L_f = 75.26 , cal / g$ con una desviación estándar de $6.35$ y un
    porcentaje de error del $5.92 \%$. Se trata de un resultado aceptable, cuyo
    error se debe a errores de medición en las masas y las temperaturas. Se podría
    decir que principalmente en la medida de la masa del hielo, ya que mientras se
    pesa y se toma la temperatura, ya ha iniciado el proceso de fusión del hielo,
    lo que reduce la masa final del hielo que será utilizada.
    \subsection*{\small Calor especifico de metales}

    Para el caso del calor especifico de los metales, se tiene que los datos
    previos a las mediciones, son los siguientes:

    \begin{itemize}
        \item $M_{cal} = 54,35 , g$
        \item $M_{agua} = 27,87 , g$
        \item $c_{agua} = 1.00 , cal/g \cdot C^{-1}$
        \item $c_{cal} = 0.22 , cal/g \cdot C^{-1}$
        \item $c_{agi} = 0.22 , cal/g \cdot C^{-1}$
    \end{itemize}

    Tras realizar el proceso descrito en la metodología, para medir el calor
    especifico de 3 metales, en principio desconocidos, pero que tras el calculo
    obtenido, se pudo determinar de que metales se trataba comparando los
    resultados con los valores consignados de distintos materiales en la
    lieteratura, se concluye que estos son aluminio, cobre y hierro. Así entonces
    se tienen los siguientes datos obtenidos del experimento, necesarios para el
    calculo. Teniendo en consideración que el experimento se realizo a una
    temperatura inicial de cada metal $T_{metal}$ al rededor de $95 , ^oC$, y la
    temperatura inicial del vaso, y del agua, $T_{0} = 26 , ^oC$. A continuación se
    presentan los datos en la tabla \ref{tab:calor_especifico}.

    \begin{table}[H]
        \centering
        \caption{Datos obtenidos para el calculo del calor especifico de los metales.}
        \begin{tabular}{cccccc}
            \hline\hline
            \textbf{$Metal$} & \textbf{$M_{agua}$} & \textbf{$M_{metal}$} & \textbf{$T_0$} & \textbf{$T_{metal}$} & \textbf{$T_e$} \\
            \small           & \small $[g]$        & \small $[g]$         & \small $[^oC]$ & \small $[^oC]$       & \small $[^oC]$ \\
            \hline\hline
            $Alummio$        & $165,16$            & $157,64$             & $26$           & $92$                 & $36$           \\
            \hline
            $Cobre$          & $159,24$            & $164,46$             & $26$           & $97$                 & $32$           \\
            \hline
            $Hierro$         & $182,8$             & $157,17$             & $26$           & $97$                 & $32$           \\
            \hline
        \end{tabular}
        \caption*{}
        \label{tab:calor_especifico}
    \end{table}

    Una vez tabulado los datos, se calcula el calor especifico de cada metal,
    utilizando la expresión \ref{eq:calor_especifico} con su correspondiente
    desviación porcentual, y se obtiene lo siguiente, tabla
    \ref{tab:calor_especifico_resultados}.

    \begin{table}[H]
        \centering
        \caption{Resultados obtenidos para el calor especifico de los metales.}
        \begin{tabular}{cccc}
            \hline\hline
            \textbf{$Metal$} & \textbf{$C$}            & \textbf{$C_{exp}$}      & \textbf{$Desviacion \, porcentual$} \\
            \small           & \small $[cal/\,g\,^oC]$ & \small $[cal/\,g\,^oC]$ &                                     \\
            \hline\hline
            $Alummio$        & $0,214$                 & $0,207$                 & $3\%$                               \\
            \hline
            $Cobre$          & $0,093$                 & $0,099$                 & $7\%$                               \\
            \hline
            $Hierro$         & $0,11$                  & $0,117$                 & $7\%$                               \\
            \hline
        \end{tabular}
        \caption*{}
        \label{tab:calor_especifico_resultados}
    \end{table}

    Se puede notar, que los valores son aceptables con un error porcentual de entre
    $3\%$ y $ 7\%$l comparado con los convencionalmente verdaderos. El error se
    atribuye a fallas de medición en las masas y las temperaturas, en especial la
    temperatura inicial del metal, ya que al momento de introducirlo al
    calorimetro, este se enfriaba rápidamente, lo cual afecta el resultado final.
    Aún así se puede concluir que los resultados representan de manera valida el
    calor especifico de los metales. Es importante resaltar que el calor especifico
    del alummio es mayor al resto debido a que este metal es menos denso que los
    demas, por lo que se requiere de mayor energía para el intercambio de calor.

    \section{\small CONCLUSIONES}
    En este experimento, se realizaron mediciones y cálculos para determinar el
    calor específico de varios metales y el calor latente de fusión del hielo. Los
    resultados obtenidos proporcionan información valiosa sobre las propiedades
    térmicas de estos materiales.\\

    En primer lugar, se llevó a cabo la determinación del calor latente del hielo
    mediante el método de fusión. Los resultados obtenidos mostraron una
    correspondencia cercana con el valor aceptado de referencia para el calor
    latente del hielo. Esto indica que el método utilizado fue efectivo para medir
    el calor latente, lo cual es un parámetro importante para caracterizar la
    transición de fase de una sustancia.\\

    Por otro lado, se determinó el calor específico de los metales mediante el
    método de calentamiento y enfriamiento. Los valores experimentales obtenidos
    para cada metal fueron consistentes con los datos teóricos de referencia. Esto
    sugiere que el método utilizado fue preciso y confiable. Además, se observó que
    el calor específico varía según el metal, lo que indica que cada uno tiene una
    capacidad única para almacenar calor.\\

    En general, se puede concluir que los resultados obtenidos tienen una precisión
    aceptable, ya que el porcentaje de error es en su mayoría inferior al $10 \%$.
    Sin embargo, se observa que para algunas medidas el porcentaje de error es
    significativamente mayor, lo que puede deberse a errores en la medición o a
    factores externos que afectaron el proceso, como se menciona, la temperatura
    inicial del metal, y la masa inicial del hielo, fueron problemas significativos
    en el experimento.\\

    \nocite{giancoli}
    \nocite{montiel2015física}

    \bibliographystyle{elsarticle-num}
    \bibliography{bib}
\end{multicols*}
\end{document}