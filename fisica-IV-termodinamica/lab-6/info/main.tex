\documentclass[10pt]{article}
\usepackage[utf8]{inputenc}
\usepackage{geometry}
\usepackage{graphicx,graphics}
\usepackage[spanish, es-tabla]{babel}
\usepackage{hyperref}
\usepackage{graphicx}
\usepackage{url}
\usepackage{float}
\usepackage{amsmath}
\usepackage{multicol}
\usepackage{amssymb}
\usepackage{multirow}
\usepackage{textcomp}
\usepackage{gensymb}
\usepackage[center]{titlesec}
\usepackage[table,xcdraw]{xcolor}

\geometry{top=2cm, bottom=2cm, left=1.5cm, right=1.5cm}
\spanishdecimal{,}
\setlength{\parskip}{0mm}

\renewcommand{\thesection}{\large\Roman{section}}
\renewcommand{\thesubsection}{\large\Alph{subsection}}

\providecommand{\abs}[1]{\lvert#1\rvert}
\title{
    \textbf TITULO
}
\author{
    \normalsize{
        \emph{Andrés F. Valencia Fonseca}$^{1}$,
        \emph{Nicolás Aguilera García}$^{2}$
    } \\
    \normalsize{
        \emph{Departamento de Física, Universidad del Valle, Cali, Colombia}
    } \\
    \small{$^{1}$2125166, $^{2}$21273030}
}
\date{(\small 13 de mayo de 2023)}

\begin{document}
\maketitle
\begin{abstract}

\end{abstract}

\begin{multicols*}{2}
    \section{\small INTRODUCCIÓN}

    \subsection*{\small Calor latente de fusión del hielo}
    Calor resultante de un proceso de cambio de temperatura: % Tener en cuenta

    \begin{equation}
        \begin{split}
            Q &= m \, c \, \Delta T \\
            &= m \, c \, (T_2 - T_1)
        \end{split}
        \label{eq:flujo_calor}
    \end{equation}

    Donde $Q$ es el flujo de energia termica (Calor), $m$ es la masa de la sustancia, $c$ es la capacidad calorifica de la sustancia y $\Delta T$ es la diferencia de temperatura entre el estado inicial y el estado final de la sustancia.

    \subsection*{\small Calor latente de fusión del hielo}
    La materia se presenta en diferentes fases, las cuales pueden variar en función de las condiciones de presión y temperatura a las que se someta. A este proceso se le denomina cambio de fase o estado (siempre y cuando no ocurra un cambio en la composición). Estos cambios de estado no ocurren de forma espontánea y requieren un aporte o liberación de energía, conocida como calor latente, que representa la energía calórica necesaria para que una sustancia cambie de estado sin que su temperatura varíe durante el proceso.

    La ecuación que modela este proceso en una sustancia es la siguiente:

    \begin{equation}
        Q_f = m \, L_f
        \label{eq:calor_latente}
    \end{equation}

    Donde $Q$ es la cantidad de calor requerida para fundir la sustancia, $m$ es la masa de la sustancia y $L_f$ es el calor latente de fusión de la sustancia.

    En este laboratorio se busca determinar el calor latente de fusión del hielo, el cual es la energía calórica requerida para que el agua ($H_2O$) cambie de fase sólida (hielo) a fase líquida.


    \section{\small METODOLOGÍA}
    Los procesos que se llevaron a cabo en este laboratorio fueron los siguientes:

    \begin{itemize}
        \item \textbf{Calor latente de fusión del hielo}

            Para determinar el calor latente de fusión, se utiliza la ecuación \ref{eq:calor_latente}, la cual requiere que se determine el calor necesario para fundir una cantidad de hielo. Para ello, se utiliza un calorímetro de masa $M'$, el cual se encuentra aislado térmicamente y cuya capacidad calórica ($c_{cal}$) es conocida. Se agrega una masa de agua $M$ con capacidad calorífica $c_{agua}$ y temperatura inicial $T_2$, que se mezcla con una masa de hielo $m$ a temperatura inicial $T_0$. Tras la mezcla, se espera un tiempo prudencial hasta que la totalidad del hielo se haya fundido completamente, y se toma la temperatura $T$ del agua, la cual se encuentra en equilibrio térmico. Con estos datos, se procede a calcular el calor latente de fusión del hielo.

            El cálculo se realiza a partir del hecho de que, dado que el calorímetro se encuentra aislado térmicamente, se puede afirmar que:
            
            \begin{equation}
                Q_{ganado} = Q_{cedido}
                \label{eq:calor_latente_1}
            \end{equation}

            Donde $Q_{ganado}$ es el flujo de calor dado por el hielo y el agua resultante del hielo fundido, mientras que $Q_{cedido}$ es el flujo de calor dado por el calorímetro y el agua inicialmente contenida en él. Si se toma en cuenta la ecuación \ref{eq:flujo_calor}, el calor ganado y cedido pueden expresarse como:

            \begin{equation}
                \begin{split}
                    Q_{ganado} &= m \, c_{agua} \, (T - 0) + m \, L_f + m \, c_{hielo} \, (0 - T_0)\\
                    Q_{cedido} &= (M \, c_{agua} + M' \, c_{cal}) \, (T_2 - T)
                \end{split}
                \label{eq:calor_latente_2}
            \end{equation}

            Al reemplazar estos valores en la ecuación \ref{eq:calor_latente_1} y despejar el valor buscado ($L_f$), se obtiene:

            \begin{equation}
                \begin{split}
                    L_f &= A - B \\
                    A &= \frac{(M \, c_{agua} + M' \, c_{cal}) \, (T_2 - T)}{m} \\
                    B &= c_{agua} \, (T - 0) + c_{hielo} \, (0 - T_0)
                \end{split}
                \label{eq:calor_latente_3}
            \end{equation}

            De esta forma, con los datos obtenidos en el laboratorio, se determina entonces el calor latente de fusión del hielo.
    \end{itemize}




    \section{\small ANÁLISIS Y RESULTADOS}
        \subsection*{\small Calor latente de fusión del hielo}
        Con los resultados obtenidos en el laboratorio, se procede a calcular el calor latente de fusión del hielo. Para ello, se toman los siguientes valores:

    \section{\small CONCLUSIONES}






    \nocite{giancoli}
    \nocite{montiel2015física}

    \bibliographystyle{elsarticle-num}
    \bibliography{bib}
\end{multicols*}
\end{document}
