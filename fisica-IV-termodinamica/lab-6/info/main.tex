\documentclass[10pt]{article}
\usepackage[utf8]{inputenc}
\usepackage{geometry}
\usepackage{graphicx,graphics}
\usepackage{hyperref}
\usepackage{graphicx}
\usepackage{url}
\usepackage{float}
\usepackage{amsmath}
\usepackage{multicol}
\usepackage{amssymb}
\usepackage{multirow}
\usepackage{textcomp}
\usepackage{gensymb}
\usepackage{hhline}
\usepackage{array}
\usepackage{caption}
\usepackage[spanish, es-tabla]{babel}
\usepackage[center]{titlesec}
\usepackage[table,xcdraw]{xcolor}

\geometry{top=2cm, bottom=2cm, left=1.5cm, right=1.5cm}
\spanishdecimal{,}
\setlength{\parskip}{0mm}

\renewcommand{\arraystretch}{1.3}
\renewcommand{\thesection}{\large\Roman{section}}
\renewcommand{\thesubsection}{\large\Alph{subsection}}

\providecommand{\abs}[1]{\lvert#1\rvert}
\title{
    \textbf TITULO
}
\author{
    \normalsize{
        \emph{Andrés F. Valencia Fonseca}$^{1}$,
        \emph{Nicolás Aguilera García}$^{2}$
    } \\
    \normalsize{
        \emph{Departamento de Física, Universidad del Valle, Cali, Colombia}
    } \\
    \small{$^{1}$2125166, $^{2}$21273030}
}
\date{(\small 13 de mayo de 2023)}

\begin{document}
\maketitle
\begin{abstract}

\end{abstract}

\begin{multicols*}{2}
    \section{\small INTRODUCCIÓN}
    \subsection*{\small Calor especifico de metales}
        Calor resultante de un proceso de cambio de temperatura: % Tener en cuenta

        \begin{equation}
            \begin{split}
                Q &= m \, c \, \Delta T \\
                &= m \, c \, (T_2 - T_1)
            \end{split}
            \label{eq:flujo_calor}
        \end{equation}

        Donde $Q$ es el flujo de energia termica (Calor), $m$ es la masa de la sustancia, $c$ es la capacidad calorifica de la sustancia y $\Delta T$ es la diferencia de temperatura entre el estado inicial y el estado final de la sustancia.


    \subsection*{\small Calor latente de fusión del hielo}
        La materia se presenta en diferentes fases, las cuales pueden variar en función de las condiciones de presión y temperatura a las que se someta. A este proceso se le denomina cambio de fase o estado (siempre y cuando no ocurra un cambio en la composición) \cite{calorLatente}. Estos cambios de estado no ocurren de forma espontánea y requieren un aporte o liberación de energía, conocida como calor latente, que representa la energía calórica necesaria para que una sustancia cambie de estado sin que su temperatura varíe durante el proceso.

        La ecuación que modela este proceso en una sustancia es la siguiente \cite{calorLatente}:

        \begin{equation}
            Q_f = m \, L_f
            \label{eq:calor_latente}
        \end{equation}

        Donde $Q$ es la cantidad de calor requerida para fundir la sustancia, $m$ es la masa de la sustancia y $L_f$ es el calor latente de fusión de la sustancia.

        En este laboratorio se busca determinar el calor latente de fusión del hielo, el cual es la energía calórica requerida para que el agua ($H_2O$) cambie de fase sólida (hielo) a fase líquida.


    \section{\small METODOLOGÍA}
    Los procesos que se llevaron a cabo en este laboratorio fueron los siguientes:

    \begin{itemize}
        \item \textbf{Calor especifico de metales:}
        
            El proceso para determinar el calor especifio de los tres diferentes metales utilizados como muestras consiste en primeramente determinar la masa de cada uno de los metales a utilizar, la masa del calorimetro y la masa del agua que estara dentro del calorimetro. Luego de esto, se procede a calentar los metales en un baño de agua hirviendo (llegando a temperaturas de $97 \, ^oC$), para posteriormente sumergirlos en el calorímetro con agua a temperatura ambiente.
            
            Tras esto, se espera a que el sistema se encuentre en equilibrio térmico, y se toma la temperatura final del sistema. Con estos datos, se procede a calcular el calor específico de cada metal mediante. Dado que el calorímetro se encuentra aislado térmicamente, se puede afirmar que en el sistema:

            \begin{equation}
                Q_{ganado} = Q_{cedido}
                \label{eq:relacion_calor}
            \end{equation}

            Donde $Q_{ganado}$ es el calor ganado por el agua y el calorímetro, y $Q_{cedido}$ es el calor cedido por el metal. El calor ganado por el agua, el calorímetro y el metal se calcula mediante la ecuación \ref{eq:flujo_calor}. Reemplazando y despejado el calor especifico del metal se obtiene la siguiente ecuación que permite realizar el calculo mediante los datos obtenidos experimentalmente:

            \begin{equation*}
                c_{m} = \frac{M_{agua} \, c_{agua} + M_{cal} \, c_{cal} + M_{agi} \, c_{agi}}{M_{metal} \, (T_{metal} - T_{e})} \cdot (T_{equ} - T_{0})
                \label{eq:calor_especifico}
            \end{equation*}

        \item \textbf{Calor latente de fusión del hielo:}

            Para determinar el calor latente de fusión, se utiliza la ecuación \ref{eq:calor_latente}, la cual requiere que se determine el calor necesario para fundir una cantidad de hielo. Para ello, se utiliza un calorímetro de masa $M'$, el cual se encuentra aislado térmicamente y cuya capacidad calórica ($c_{cal}$) es conocida. Se agrega una masa de agua $M$ con capacidad calorífica $c_{agua}$ y temperatura inicial $T_2$, que se mezcla con una masa de hielo $m$ a temperatura inicial $T_0$. Tras la mezcla, se espera un tiempo prudencial hasta que la totalidad del hielo se haya fundido completamente, y se toma la temperatura $T$ del agua, la cual se encuentra en equilibrio térmico. Con estos datos, se procede a calcular el calor latente de fusión del hielo.

            El calculo al igual que en el caso anterior, se realiza mediante la ecuación \ref{eq:relacion_calor}, pero teniendo en cuenta que el calor ganado y cedido son diferentes.

            Para este caso, el $Q_{ganado}$ es el flujo de calor dado por el hielo y el agua resultante del hielo fundido, mientras que $Q_{cedido}$ es el flujo de calor dado por el calorímetro y el agua inicialmente contenida en él. Si se toma en cuenta la ecuación \ref{eq:flujo_calor}, el calor ganado y cedido pueden expresarse como:

            \begin{equation}
                \begin{split}
                    Q_{ganado} &= m \, c_{agua} \, (T - 0) + m \, L_f + m \, c_{hielo} \, (0 - T_0)\\
                    Q_{cedido} &= (M \, c_{agua} + M' \, c_{cal}) \, (T_2 - T)
                \end{split}
                \label{eq:calor_latente_2}
            \end{equation}

            Al reemplazar estos valores en la ecuación \ref{eq:calor_latente_1} y despejar el valor buscado ($L_f$), se obtiene:

            \begin{equation}
                \begin{split}
                    L_f &= A - B \\
                    A &= \frac{(M \, c_{agua} + M' \, c_{cal}) \, (T_2 - T)}{m} \\
                    B &= c_{agua} \, (T - 0) + c_{hielo} \, (0 - T_0)
                \end{split}
                \label{eq:calor_latente_3}
            \end{equation}

            De esta forma, con los datos obtenidos en el laboratorio, se determina entonces el calor latente de fusión del hielo.
    \end{itemize}




    \section{\small ANÁLISIS Y RESULTADOS}
        \subsection*{\small Calor latente de fusión del hielo}
        Con los resultados obtenidos en el laboratorio, se procede a calcular el calor latente de fusión del hielo. Para esto, se tienen además los siguientes datos:

        \begin{itemize}
            \item $M' = 92.8 , g$
            \item $c_{agua} = 1.00 , cal/g \cdot C^{-1}$
            \item $c_{cal} = 0.22 , cal/g \cdot C^{-1}$
            \item $c_{hielo} = 0.50 , cal/g \cdot C^{-1}$
        \end{itemize}

        Los resultados se obtuvieron al realizar el proceso descrito en la metodología para tres masas de hielo diferentes, donde para cada masa se tomó una medida a alta temperatura (en un rango entre $37 , ^oC$ y $39 , ^oC$) y una a temperatura ambiente ($34 , ^oC$), obteniendo un total de 6 medidas. Los resultados obtenidos se muestran en la Tabla \ref{tab:calor_latente}.

        \begin{table}[H]
            \centering
            \caption{Resultados obtenidos para el calor latente de fusión del hielo.}
            \begin{tabular}{ccccccc}
                \hline\hline
                \textbf{$M$} & \textbf{$m$} & \textbf{$T_2$} & \textbf{$T_0$} & \textbf{$T$} & \textbf{$L_f$} & \textbf{$\% Error$} \\
                \small $[g]$ & \small $[g]$ & \small $[^oC]$ & \small $[^oC]$ & \small $[^oC]$ & \small $[cal / g]$ &  \\
                \hline\hline
                $147.75$ & $14.31$ & $37$ & $7$ & $28$ & $81.26$ & $1.58$ \\
                $163.13$ & $14.62$ & $24$ & $7$ & $17$ & $74.38$ & $7.02$ \\
                \hline
                $182.91$ & $6.11$ & $39$ & $7$ & $36$ & $67.33$ & $15.83$ \\
                $177.08$ & $6.85$ & $24$ & $7$ & $21$ & $68.99$ & $13.76$ \\
                \hline
                $161.00$ & $5.04$ & $38$ & $7$ & $35$ & $76.49$ & $4.39$ \\
                $224.65$ & $4.80$ & $24$ & $6$ & $22$ & $83.11$ & $3.89$ \\
                \hline
            \end{tabular}
            \caption*{\textit{Nota:} El porcentaje de error se calculo respecto al valor teórico del calor latente de fusión del hielo, el cual es $80 \, cal / g$ dado en la guia de laboratorio.}
            \label{tab:calor_latente}
        \end{table}

        De estos resultados se puede observar que no existe una relación entre la certeza del resultado y la temperatura inicial utilizada para el agua. Si se revisan los resultados por las masas utilizadas, se obtiene lo siguiente:

        \begin{itemize}
        \item $m \approx 14.6 , g$: Desviación estándar de $4.87$, el valor promedio es de $L_f = 77.82 , cal / g$ y un porcentaje de error del $2.72 \%$.
        \item $m \approx 6.8 , g$: Desviación estándar de $1.17$, el valor promedio es de $L_f = 68.16 , cal / g$ y un porcentaje de error del $14.79 \%$.
        \item $m \approx 5.0 , g$: Desviación estándar de $4.68$, el valor promedio es de $L_f = 79.80 , cal / g$ y un porcentaje de error del $0.25 \%$.
        \end{itemize}

        Se puede observar que los resultados concuerdan con el valor teórico, siendo bastante más preciso el obtenido para $m \approx 5.0 , g$. Sin embargo, al igual que en el caso de la temperatura, no se observa una relación entre la masa y la certeza del resultado.

        En cuanto al valor promedio del calor latente de fusión del hielo, se obtiene que es de $L_f = 75.26 , cal / g$ con una desviación estándar de $6.35$ y un porcentaje de error del $5.92 \%$. Se trata de un resultado aceptable, cuyo error se debe a errores de medición en las masas y las temperaturas. Se podría decir que principalmente en la medida de la masa del hielo, ya que mientras se pesa y se toma la temperatura, ya ha iniciado el proceso de fusión del hielo, lo que reduce la masa final del hielo que será utilizada.

    
    \section{\small CONCLUSIONES}
    En general, se puede concluir que los resultados obtenidos tienen una precisión aceptable, ya que el porcentaje de error es en su mayoría inferior al $10 \%$. Sin embargo, se observa que para algunas medidas el porcentaje de error es significativamente mayor, lo que puede deberse a errores en la medición o a factores externos que afectaron el proceso. 






    \nocite{giancoli}
    \nocite{montiel2015física}

    \bibliographystyle{elsarticle-num}
    \bibliography{bib}
\end{multicols*}
\end{document}
